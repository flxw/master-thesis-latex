% Abstract

%\renewcommand{\abstractname}{Abstract} % Uncomment to change the name of the abstract

\pdfbookmark[1]{Abstract}{Abstract} % Bookmark name visible in a PDF viewer

\begingroup
\let\clearpage\relax
\let\cleardoublepage\relax
\let\cleardoublepage\relax

\chapter*{Abstract}
Manual labor processes have seen a lot of automation in the past, but knowledge-intensive processes still proceed in a highly manual fashion.
Systems to assist knowledge workers have been asked for in recent literature reviews.
Such assistance systems need to anticipate the development of a case to offer assistance in the right circumstances.
This includes the capability to foresee the next activity in a process.
Predicting the next activity in a running process is a young discipline called Predictive Process Monitoring, which we contribute to by evaluating neural network prediction models for this task.

In this work, we connect next-activity prediction to sequence prediction.
Thanks to this connection, we can adapt a prediction model from a natural language processing competition where the next word in a sentence was to be predicted.
Additionally, we augment this prediction model with a different feature set to produce a second approach.
To have a direct comparison, we reimplement two published prediction models for business processes.
Because several strategies are common to create batches from sequential data of variable length, we include a comparison of these strategies in our evaluation.
To make model training easier and facilitate the inclusion of the strategies into the training process, we also propose a model training framework.
The framework allowed easy training of the four models with four different batch creation strategies on eight datasets.

In the evaluation, we note a connection between process complexity and prediction accuracy.
Furthermore, we realize that one batching strategy delivers the most promising results, and should be explored further.
Finally, we compare our accuracies with numbers from three recent next-activity prediction approaches on two different real-life process event logs.
All of our four models outperform these numbers.
\newpage
\pdfbookmark[1]{German Abstract}{German Abstract}
\chapter*{German Abstract}
Wissensarbeit läuft größtenteils manuell ab.
Assistenzsysteme mit unterstützender Funktion für Wissensarbeiter werden daher häufig gefordert.
Diese Systeme müssen die Entwicklung eines Prozesses vorhersehen können, um effektiv zu sein.
Dazu zählt auch, die unmittelbar nächste Prozessaktivität zu antizipieren.
Just diese Fähigkeit versuchen wir mittels neuronaler Netze zu verbessern.

Wir verknüpfen die Vorhersage von Prozessaktivitäten mit der Vorhersage von Sequenzen, und können so ein auf Sätzen erprobtes neuronales Netzwerk anpassen.
Zusätzlich erstellen wir eine zweite Version des Netzes, die eine angepasste Form von Daten verarbeitet.
Weiterhin reproduzieren wir zwei bereits veröffentlichte Netzwerke zur Vorhersage der nächsten Prozessaktivität, um eine Vergleichsbasis zu schaffen.
Während der Implementierung fielen verschiedene Strategien auf, aus sequentiellen Daten verschiedener Länge Batches zum Training neuronaler Netzwerke zu schaffen.
Wir möchten zum Verständnis dieser Strategien beitragen, und beziehen ein Vergleich in die Arbeit ein.
Um das Trainieren der Netzwerke mit den verschiedenen Strategien zu vereinfachen, präsentieren wir im Laufe der Arbeit ein Trainings-Framework.
Mittels dieses Frameworks können wir vier verschiedene Netzwerke auf Basis vier verschiedener Strategien auf acht verschiedenen Datensätzen trainieren.

In der Auswertung stellen wir einen Zusammenhang zwischen Komplexität und Genaugikeit der Vorhersagen fest.
Des weiteren stellt sich eine der Batch-Erstellungsstrategien als besonders gut heraus, und sollte erweitert werden.
Abschließend vergleichen wir unsere Ergebnisse mit drei Arbeiten auf Basis zwei Prozess Logs.
Auf beiden Logs erreichen alle unserer vier Modelle eine deutlich höhere Genauigkeit.

\endgroup

\vfill
