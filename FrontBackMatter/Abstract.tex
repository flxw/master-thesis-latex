% Abstract

%\renewcommand{\abstractname}{Abstract} % Uncomment to change the name of the abstract

\pdfbookmark[1]{Abstract}{Abstract} % Bookmark name visible in a PDF viewer

\begingroup
\let\clearpage\relax
\let\cleardoublepage\relax
\let\cleardoublepage\relax

\chapter*{Abstract}
Manual labor processes have seen a lot of automation in the past, but knowledge-intensive processes still proceed highly manual.
Assistance systems for knowledge work have been asked for in recent several literature reviews.
Such systems need to anticipate the development of a case in order to offer assitance in the right circumstances.
This capability includes the ability to foresee the next activity in a process.
Predicting the next activity in a running process is a young discipline which we aimed to contribute to.

We do so by connecting process prediction to sequence prediction, and adapting a successfully tested word prediction model.
Additionally, we augmented it with a different feature set in a second model.
We implement two published models as baselines to avoid comparability issues due to data preprocessing.
To make model training easier and increase comparability, we also propose a model training framework.
Using the framework, we evaluate four models with four preprocessing methods on eight datasets.

Finally, we compare the results to two recent next-activity prediction approaches on a log from BPIC 2012 and on the HelpDesk log.
On the former, we are able to obtain an accuracy of $0.853$, outperforming the compared approaches by approximately $0.07$.
On the latter log, we reach an accuracy of $0.862$.

\pdfbookmark[1]{German Abstract}{German Abstract}
\chapter*{German Abstract}
Wissensarbeit läuft größtenteils manuell ab.
Assistenzsysteme mit unterstützender Funktion für Wissensarbeiter wurden bereits häufig gefordert.
Diese Systeme müssen die Entwicklung eines Prozesses vorhersehen können, um effektiv zu sein.
Dazu zählt auch, die unmittelbar nächste Prozessaktivität zu antizipieren.
Just diese Fähigkeit steht im Mittelpunkt unserer Arbeit.
Wir verknüpfen die Vorhersage von Prozessaktivitäten mit der Vorhersage von Sequenzen, und können so ein auf Sätzen erprobtes neuronales Netzwerk anpassen.
Zusätzlich erstellen wir eine Version die eine angepasste Form von Daten verarbeitet.
Weiterhin bauen wir zwei veröffentlichte Netzwerke nach, um eine Vergleichsbasis zu schaffen.
Um das Trainieren solcher Netzwerke zu vereinfachen, präsentieren wir zudem ein Trainings-Framework.
%Mittels dieses Frameworks können wir rasch vier verschiedene Netzwerke auf Basis vier verschiedener Vorverarbeitungsmechanismen auf acht verschiedenen Datensätzen trainieren.

Wir vergleichen unsere Ergebnisse mit drei Arbeiten auf Basis des BPIC 2012 Logs und des HelpDesk Logs.
Auf dem BPIC 2012 Log erreichen unsere Modelle eine Genauigkeit von $0.853$, und auf dem HelpDesk Log eine Genauigkeit von $0.862$.

\endgroup

\vfill
