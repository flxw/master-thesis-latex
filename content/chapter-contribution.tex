\chapter{Contribution}\label{sec:contribution}
This thesis shall be understood as an extension to the works over Evermann et al. \cite{evermann2016} and Schönig et al. \cite{schoenig2018}, both of which were presented in the previous section.

The two works have demonstrated the applicability of LSTM neural networks in Predictive Process Monitoring, but have left out the general perspective on the sequence prediction problem. Furthermore, the impact of added features to the training data was not investigated.

This work shall introduce the general perspective through the adaption of the approach of Shibata et al. \cite{shibata2016bipartite}. Their bipartite network architecture as well as the engineered SP2 features have shown extraordinary performance in the SPiCE competition \cite{web:spice}.

XXX hint at using subsequential information

Do this with PrefixSpan

shibata presents his SP2 approach

I want to build on those works and use subsequentially mined features with Shibata's network architecture as well as his style

\section{Understanding a trace as a sequence}

\section{Adapting s network architecture}
