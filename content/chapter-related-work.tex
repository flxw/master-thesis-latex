\chapter{Related Work}\label{chap:related-work}

\cleanchapterquote{A picture is worth a thousand words. An interface is worth a thousand pictures.}{Ben Shneiderman}{(Professor for Computer Science)}


Hauder et al. mention numerous research challenges in the domain of ACM, among them an active support system for knowledge workers \cite{hauder2014}.
The need for such a system is emphasized by Francescomarino et al. in their literature review, where it has been found that few prediction approaches target the next activity \cite{francescomarino2018}.

An example for how such a system might look like is given by Huber, who has developed a next-step recommendation system serving different case goals.
The system is prototypically implemented into CoCaMa\footnote{CoCaMa is an abbreviation for a project called Collaborative Case Management, which appears to be retired: \url{http://archive.li/uZFnN}}, a prototypical case management application. The system has been evaluated with 25 hand-made case logs.

Building upon each other are the works by Evermann et al. \cite{evermann2016} and Schönig et al. \cite{schoenig2018}.
Evermann et al. have successfully demonstrated the good performance of long-short-term memory (LSTM) neural networks in predicting the next activity.
Their approach did not take into account specific case data attributes however.
How making use of this contextual information can improve the prediction accuracy even more, has been shown by Schönig et al. \cite{schoenig2018}.
Furthermore Schönig et al. have explored data preparation methods for supporting the model during learning.

Similarly, Polato et al. make use of environmental information in their work for improving the prediction of the remaining time of business process instances \cite{polato2014}.

Metzger et al. predict run-time of a case by comparing and combining different prediction models into a model ensemble.
Then, the members of the ensemble are selected based on their predictive performance measures.
This allows taking into account costs of false predictions \cite{metzger2015}.

Francescomarino et al. have performed clustering in the preprocessing phase of model training and prediction.
Having clustered the training data, one model was created and trained for each cluster.
For obtaining a prediction, the optimal cluster for a new data item is found from which the corresponding model is selected.
This approach was evaluated on the accuracy of predicate fulfillment with two different clustering methods (k-means and DBSCAN) and two different prediction models (decision trees and random forests) \cite{francescomarino2015}.
A further evaluation criteria was \textit{earliness}, i.e. at which point in time the correct result could be determined.

\todo[inline]{SPICE competition submissions}
\todo[inline]{LSTM application on sequences}
\todo[inline]{LSTM application on sequences}
\todo[inline]{PrefixSpan, Strictly-Piecewise features...}
