\chapter{Evaluation}\label{chap:evaluation}
The contribution of this thesis is the direct comparison of the network architectures presented in the previous section on data from the Business Process Intelligence Competitions 2011 and 2012~\cite{BPIC2011, BPIC2012}. The execution of this comparison in code, the preceding data preparation and transformation as well as the obtained results shall be presented in this chapter.

\section{Implementation}
Anaconda~\cite{web:anaconda} was set up to create a stable working environment in which dependencies and libraries would not change.
Then, the data was explored, prepared and cleansed in an iterative fashion using JupyterLab notebooks~\cite{web:jupyter}. Helpful libraries in the process were prefixspan-py~\cite{web:prefixspan-py} and OpyenXes~\cite{web:opyenxes}. These notebooks were also used to develop the ANN training code in Keras~\cite{web:keras}.

As the codebase became more mature, Docker containers were built with the Anaconda environment inside them~\cite{web:docker}. Using a version of Docker for GPU applications running on NVIDIA hardware~\cite{web:nvidia-docker}, the networks were trained on the GPU cluster infrastructure of the HPI FutureSOC Lab~\cite{web:fsoc}.
The source code used in this thesis is available for download at: \url{https://github.com/flxw/master-thesis-code}.

\section{Data preprocessing}
Following the KDD process~\cite{fayyad1996data}, the data was preprocessed to eliminate generally known properties that hinder machine learning model performance. Here, this encompassed two steps for both datasets:

\begin{enumerate}
    \item All columns which exhibited zero entropy, i.e. were constituted of a single value, were dropped
    \item Using the bias-corrected version of Cramér's~V~\cite{bergsma2013bias}, features that correlate strongly were dropped
\end{enumerate}


For the BPIC2011 dataset, the column \texttt{lifecycle:transition} was dropped in step 1. In step 2, 
activity code correlates to 

\section{Data transformation}
one-hot encoding for certain columns since no ordinal relationship present
Which columns were engineered and how
Engineering of SP2 features and PrefixSpan \cite{web:prefixspan-py}

\begin{lstlisting}
def enrich_trace_with_sp2(t):
    sp2_df = pd.DataFrame(columns=activity_labels, index=range(0,len(t)), dtype=np.bool)
    for col in sp2_df.columns: sp2_df[col].values[:] = 0
    sp2_df["{0}{1}".format(sp2_prefix, t["concept:name"][0])].values[0]  = 1
    
    for i in range(1,len(t)):
        first_activity_name = t["concept:name"].iloc[i]
        col = "{0}{1}".format(sp2_prefix,first_activity_name)
        
        sp2_df.values[i] = sp2_df.values[i-1]
        sp2_df[col].values[i] = 1
        
    return sp2_df
\end{lstlisting}

\section{Test setup}
Dropping the embedding layer if results are really bad
Iterative way of working
Making embedding layer optional
How batches, timesteps and samples were framed for this problem

\section{Results}
Three metrics of the four aforementioned implementations shall be the focus of the evaluation of the results:
\begin{enumerate}
    \item\textbf{Accuracy} - I.e. how many next-activity predictions were correct in total.
    \item\textbf{Earliness} - I.e. whether and how early in the total trace the prediction accuracy begins to stabilize \cite{francescomarino2015}.
    \item\textbf{Feature importance} - I.e. whether network-internal metrics indicate that an added feature is valuable for making predictions.
\end{enumerate}

Accuracy and earliness

Does accuracy rise with increasing trace length?
Feature Importance score extraction

\section{Discussion}
Is the hassle worth it?
Increase in memory, training time...?
Compare with Evermanns etc results