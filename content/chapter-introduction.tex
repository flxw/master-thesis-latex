\chapter{Introduction}\label{sec:intro}
\setlength{\epigraphwidth}{0.9\textwidth}
\renewcommand{\epigraphrule}{0pt}
\epigraph{\textit{The most important contribution of management in the 20th century was to increase manual worker productivity fifty-fold. The most important contribution of management in the 21st century will be to increase knowledge worker productivity—hopefully by the same percentage}}{Peter Drucker in 1999 -- CHECK THIS YEAR}

\lipsum[1-4]

\section{Research questions}\label{sec:intro:objective}
\todo[inline]{Does including subsequence information lead to improved predictions?}
\todo[inline]{Does more data lead to better predictions?}

In my thesis, I want to investigate the synergies of combining the aforementioned approach of Francescomarino et al. for learning data clustering with LSTM neural networks as per Evermann et al. Case data attributes shall be used during model training and prediction, as Polato et al. \cite{polato2014} and Schönig et al. \cite{schoenig2018} demonstrated their usefulness.

This would contribute to a field of research which is currently being explored and where LSTM networks have been applied successfully on prediction problems with long-term dependencies \cite{evermann2016, tax2017, schoenig2018, graves2005}.

%Furthermore, I want to investigate how much the accuracy of current LSTM approaches is improved if the learning data is clustered.
Furthermore, I want to determine how historical case log data is prepared best for learning, as only Schönig et al. has written a small subsection on this \cite{schoenig2018}.
If time permits, I also want to investigate the potential of ensembles within this context, as they can potentially enlighten the user about the reason for a prediction.
With neural networks it is hard to comprehend the reasons behind a prediction.
Other types of models deliver better comprehensibility.

Throughout the document I will strive to meet recently demanded machine learning paper quality criteria \cite{lipton2018}.

The performance of the combined approaches shall be evaluated against the data from the Business Process Intelligence Challenges (BPIC) 2011, 2012 and 2017 \cite{BPIC2011, BPIC2012, BPIC2017}. This allows for comparison with the results of Francescomarino et al., Evermann et al., Tax et al. and Schönig et al. \cite{francescomarino2018, evermann2016, tax2017, schoenig2018}.
The next steps and an approximate timeframe are shown in the table below:\\[1em]

\todo[inline]{taylorism}

\textbf{Does including subsequence information improve prediction quality}

\textbf{Does more data lead to better predictions? If so, which criteria must it present?}

Adaptive Case Management (ACM)

1. establish baseline mimicking evermann and schönig
enhance with prefixspan-mined features and sp-k encoded features
measure difference
measure earliness

Three pillars to my work
Evermann
Schönig
Shibata

Combination
Publish benchmark in comparison to other precisions and approaches

\section{Thesis Structure}\label{sec:intro:structure}


